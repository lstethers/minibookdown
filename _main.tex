% Options for packages loaded elsewhere
\PassOptionsToPackage{unicode}{hyperref}
\PassOptionsToPackage{hyphens}{url}
%
\documentclass[
  openany]{book}
\usepackage{lmodern}
\usepackage{amssymb,amsmath}
\usepackage{ifxetex,ifluatex}
\ifnum 0\ifxetex 1\fi\ifluatex 1\fi=0 % if pdftex
  \usepackage[T1]{fontenc}
  \usepackage[utf8]{inputenc}
  \usepackage{textcomp} % provide euro and other symbols
\else % if luatex or xetex
  \usepackage{unicode-math}
  \defaultfontfeatures{Scale=MatchLowercase}
  \defaultfontfeatures[\rmfamily]{Ligatures=TeX,Scale=1}
\fi
% Use upquote if available, for straight quotes in verbatim environments
\IfFileExists{upquote.sty}{\usepackage{upquote}}{}
\IfFileExists{microtype.sty}{% use microtype if available
  \usepackage[]{microtype}
  \UseMicrotypeSet[protrusion]{basicmath} % disable protrusion for tt fonts
}{}
\makeatletter
\@ifundefined{KOMAClassName}{% if non-KOMA class
  \IfFileExists{parskip.sty}{%
    \usepackage{parskip}
  }{% else
    \setlength{\parindent}{0pt}
    \setlength{\parskip}{6pt plus 2pt minus 1pt}}
}{% if KOMA class
  \KOMAoptions{parskip=half}}
\makeatother
\usepackage{xcolor}
\IfFileExists{xurl.sty}{\usepackage{xurl}}{} % add URL line breaks if available
\IfFileExists{bookmark.sty}{\usepackage{bookmark}}{\usepackage{hyperref}}
\hypersetup{
  pdftitle={Bookdown example with Actions Workflow},
  pdfauthor={Lori Stethers},
  hidelinks,
  pdfcreator={LaTeX via pandoc}}
\urlstyle{same} % disable monospaced font for URLs
\usepackage{longtable,booktabs}
% Correct order of tables after \paragraph or \subparagraph
\usepackage{etoolbox}
\makeatletter
\patchcmd\longtable{\par}{\if@noskipsec\mbox{}\fi\par}{}{}
\makeatother
% Allow footnotes in longtable head/foot
\IfFileExists{footnotehyper.sty}{\usepackage{footnotehyper}}{\usepackage{footnote}}
\makesavenoteenv{longtable}
\usepackage{graphicx,grffile}
\makeatletter
\def\maxwidth{\ifdim\Gin@nat@width>\linewidth\linewidth\else\Gin@nat@width\fi}
\def\maxheight{\ifdim\Gin@nat@height>\textheight\textheight\else\Gin@nat@height\fi}
\makeatother
% Scale images if necessary, so that they will not overflow the page
% margins by default, and it is still possible to overwrite the defaults
% using explicit options in \includegraphics[width, height, ...]{}
\setkeys{Gin}{width=\maxwidth,height=\maxheight,keepaspectratio}
% Set default figure placement to htbp
\makeatletter
\def\fps@figure{htbp}
\makeatother
\setlength{\emergencystretch}{3em} % prevent overfull lines
\providecommand{\tightlist}{%
  \setlength{\itemsep}{0pt}\setlength{\parskip}{0pt}}
\setcounter{secnumdepth}{5}

\title{Bookdown example with Actions Workflow}
\author{Lori Stethers}
\date{}

\begin{document}
\maketitle

{
\setcounter{tocdepth}{1}
\tableofcontents
}
\hypertarget{im-creating-a-sample-textbook}{%
\chapter{I'm creating a sample textbook}\label{im-creating-a-sample-textbook}}

We are getting started with Bookdown, Markdown, and Actions

\hypertarget{things-id-like-do-with-actions}{%
\chapter{Things I'd like do with Actions}\label{things-id-like-do-with-actions}}

\textbf{1. Make file available for lib staff}: Push a file to Google Drive each time it is committed. I've found these actions that may do what I want:
* \url{https://github.com/satackey/action-google-drive}
* \url{https://github.com/marketplace/actions/upload-files-to-google-drive}
* \url{https://github.com/Jodebu/upload-to-drive}

\textbf{2. Test XSLT}: Run an XSLT file against a directory of XML files every time the XSLT file is committed. Send the resulting formatted output/errors to either email, to the screen, or to a folder so I can review them.
Normally I test my XSLT manually using XMLNotepad against several XML files with data representing different scenarios, but it would be so much faster to have this done automatically and then save or email me the output.
I found this (xslt-action){[}\url{https://github.com/gvlx/xslt-action}{]} repository, but I don't think it's a GitHub action, I think it's specifically for testing inside Visual Studio.
I can't find any examples of how I might achieve this with GitHub Actions.

\textbf{3. Code management}: In github I will be managing scripts and web forms that send emails to people. When I am testing them I generally modify the code to send all the emails to a test account or to myself. Then I modify it back to email the real users when I roll it into production. Is this code change for testing vs.~prod something I could do automatically through Actions? Or some other way in github, like branching? I'm not sure how to approach it.

\textbf{4. Test web forms}: Spin up a web server so I can test my php web forms on it every time they are committed (lower priority since I'm hoping we'll be retiring these forms in the next year)

\hypertarget{projects-id-like-to-put-in-github}{%
\chapter{Projects I'd like to put in GitHub}\label{projects-id-like-to-put-in-github}}

\begin{itemize}
\tightlist
\item[$\boxtimes$]
  Illiad website
\item[$\boxtimes$]
  Islandora MODS XML to Marc XML transformation file
\item[$\square$]
  Alma letters
\item[$\square$]
  PrimoVE Discovery Import normalization rules
\item[$\square$]
  Locally customized Illiad Add-ons
\item[$\square$]
  PrimoVE web configuration files
\item[$\square$]
  code for our library Moodle block
\item[$\square$]
  Random other scripts: alma-browzine-export.py, ill\_twiliosmsforward.php, past perl and shell scripts that I might need again
\item[$\square$]
  Course reserve request submission forms (low priority, hopefully these will go away first)
\end{itemize}

\end{document}
